\documentclass[captions=tableheading,parskip=half, bibliography=totoc]{scrartcl} % KOMA-Script Dokumentenklasse Article [titlepage=firstiscover]

% UTF8 input encoding support
\usepackage[utf8]{inputenc}

% scrhack
\usepackage{scrhack}

% Warnung, falls noch einmal kompiliert werden muss
\usepackage[aux]{rerunfilecheck}

% Paket für Schriftarteinstellung, muss immer geladen werden
\usepackage{fontspec}

% Deutsche Spracheinstellungen, wichtig z. B. für korrekte Trennung
\usepackage[main=nenglish]{babel}
\usepackage[autostyle]{csquotes}

\usepackage{ragged2e}
\usepackage{pdfpages}
\usepackage{pdflscape}
\usepackage{rotating}

% adds Blindtext
\usepackage{blindtext}
\usepackage{longtable}

% Math and Science extensions
\usepackage{amsmath, nccmath}
\usepackage{amssymb}
\usepackage{mathtools}
\DeclarePairedDelimiter\abs{\lvert \,}{\rvert}
\DeclarePairedDelimiter{\bra}{\langle}{\rvert}
\DeclarePairedDelimiter{\ket}{\lvert}{\rangle}
\DeclarePairedDelimiterX{\braket}[2]{\langle}{\rangle}{
    #1 \delimsize| #2
    }

\usepackage[math-style=ISO,bold-style=ISO,sans-style=italic,nabla=upright,partial=upright]{unicode-math}
\usepackage[locale=DE, uncertainty-mode = separate]{siunitx}  %[per-mode=symbol-or-fraction]
\sisetup{math-micro=\text{$\symup{\mu}$},text-micro=$\symup{\mu}$}
\usepackage[version=4, math-greek=default, text-greek=default]{mhchem}
\DeclareSIUnit\Div{DIV}
%
\usepackage{graphicx}
\usepackage{grffile}
\usepackage{float}
\floatplacement{figure}{htbp}
\usepackage[labelfont=bf]{caption}
\usepackage[width=0.75\textwidth]{subcaption}

% Bibliography extensions
\usepackage{booktabs}
\usepackage[backend=biber]{biblatex}
\addbibresource{../lit.bib}

\usepackage{microtype}
\usepackage{xfrac}
\usepackage[shortcuts]{extdash}
\usepackage{expl3}
\usepackage{xparse}
\usepackage{mleftright}
\usepackage[shortcuts]{extdash}

\NewDocumentCommand \delnachdel {mm}
{
    \mathinner{\frac{\partial #1}{\partial #2}}
}
\NewDocumentCommand \delnachdelsq {mm}
{
    \mathinner{\frac{\partial^2 #1}{\partial #2 ^2}}
}
\NewDocumentCommand \dnachd {mm}
{
    \mathinner{\frac{\symup{d} #1}{\symup{d} #2}}
}

\NewDocumentCommand \deln {mmm}
{
    \mathinner{\frac{\partial^#3 #1}{\partial #2 ^#3}}
}

\NewDocumentCommand \leftside {m}
{
\flushleft{#1\;}\justifying
}


\NewDocumentCommand \dif {m}
{
    \,\mathinner{\symup{d} #1}
}

\NewDocumentCommand \zz {}
{
    \mathinner{\mathrm{Z\kern-.3em\raise-0.5ex\hbox{Z}}}
}

\ExplSyntaxOff



% Unterstützung für Links und PDF Metadaten
\usepackage[unicode,pdfusetitle]{hyperref}
\usepackage{bookmark}

%\setmathfont{XITS Math}[range={scr,bfcal}]
%\setmathfont{XITS Math}[range={cal,bfcal}, StylisticSet=1]
\sisetup{math-micro=\text{\mu},text-micro=\mu}

\begin{document}
    \pagenumbering{roman}

    \vspace{2cm}
    
    \title{Characterization of Scintilating Fibres}
    
    \vspace{1cm}
    
    \author{
        Luca Di Bella\\
        \texorpdfstring{\href{mailto:luca.dibella@tu-dortmund.de}{luca.dibella@tu-dortmund.de}\and}{,}
        Luca Fiedler\\
        \texorpdfstring{\href{mailto:luca.fiedler@tu-dortmund.de}{luca.fiedler@tu-dortmund.de}}{}
    }
    
    \vspace{1cm}
    
    \maketitle
    \thispagestyle{empty}
    
    \vfill
    
    \begin{center}
        Technische Universität Dortmund\\
        Advanced Particle Physics Lab Course
    \end{center}
    
    \newpage
    \justifying
    \tableofcontents
    \newpage
    \pagenumbering{arabic}
    \clearpage
    \setcounter{page}{1}

% -- Zielsetzung -- %
\section{Introduction}
    Scintilators or in general scintilating materials are often used in particle physics experiments for their light emitting properties when charged particles enter and propagate trough them.
    Usecases range from detecting if there are any particles to measuring the halflife of cosmic radiation by measuring the light with photomultipliers.
    In this lab course a scintilating fibre and its light emission is analysed as it is part of a larger detector named SciFi-Tracker.
\subsection{SciFi and LHCb}
    The Large Hadron Collider (LHC) is the largest Collider consisting of many detectors working to measure the processes of proton-proton collisions in the $\si{\teraelectronvolt}$ range.
    Individual detector layers work together to measure every critical aspect of the several orders of interaction prodcts such as charge, impuls, energy for example.
    SciFi's role in this is to measure the coordinate and direction of created charged particles in the moments after the one collision. 
    To do this a fabric of many $\SI{250}{\micro\metre}$ thick is constructed and organized in four layers forming a x-u-v-x-patters where u and v are angled at $+ 5 \degree$ and $- 5 \degree$ respectively.
    Three of these segments with spacing in between make up the scintilating part of SciFi.

    The individual fibres have a core layer of scintilating polysterol and two coatings of material with decreasing optical index.
    With the coating total internal reflection is achieved allowing the light of a passing particle to travel only to one of the two ends of the fibre.
    One end is mirrored and the other is a silicone photomultiplier (SiPM) to convert the optical pulse into an electrical signal.

    %
    % -- Theorie -- %
    %
\section{Theory}

%
% -- Auswertung -- %
%
  

%
% -- Diskussion -- %
%


\newpage
\printbibliography
%\newpage
\end{document}