\documentclass[captions=tableheading,parskip=half, bibliography=totoc]{scrartcl} % KOMA-Script Dokumentenklasse Article [titlepage=firstiscover]

% UTF8 input encoding support
\usepackage[utf8]{inputenc}

% scrhack
\usepackage{scrhack}

% Warnung, falls noch einmal kompiliert werden muss
\usepackage[aux]{rerunfilecheck}

% Paket für Schriftarteinstellung, muss immer geladen werden
\usepackage{fontspec}

% Deutsche Spracheinstellungen, wichtig z. B. für korrekte Trennung
\usepackage[main=nenglish]{babel}
\usepackage[autostyle]{csquotes}

\usepackage{ragged2e}
\usepackage{pdfpages}
\usepackage{pdflscape}
\usepackage{rotating}

% adds Blindtext
\usepackage{blindtext}
\usepackage{longtable}

% Math and Science extensions
\usepackage{amsmath, nccmath}
\usepackage{amssymb}
\usepackage{mathtools}
\DeclarePairedDelimiter\abs{\lvert \,}{\rvert}
\DeclarePairedDelimiter{\bra}{\langle}{\rvert}
\DeclarePairedDelimiter{\ket}{\lvert}{\rangle}
\DeclarePairedDelimiterX{\braket}[2]{\langle}{\rangle}{
    #1 \delimsize| #2
    }

\usepackage[math-style=ISO,bold-style=ISO,sans-style=italic,nabla=upright,partial=upright]{unicode-math}
\usepackage[locale=DE, uncertainty-mode = separate]{siunitx}  %[per-mode=symbol-or-fraction]
\sisetup{math-micro=\text{$\symup{\mu}$},text-micro=$\symup{\mu}$}
\usepackage[version=4, math-greek=default, text-greek=default]{mhchem}
\DeclareSIUnit\Div{DIV}
%
\usepackage{graphicx}
\usepackage{grffile}
\usepackage{float}
\floatplacement{figure}{htbp}
\usepackage[labelfont=bf]{caption}
\usepackage[width=0.75\textwidth]{subcaption}

% Bibliography extensions
\usepackage{booktabs}
\usepackage[backend=biber]{biblatex}
\addbibresource{../lit.bib}

\usepackage{microtype}
\usepackage{xfrac}
\usepackage[shortcuts]{extdash}
\usepackage{expl3}
\usepackage{xparse}
\usepackage{mleftright}
\usepackage[shortcuts]{extdash}

\NewDocumentCommand \delnachdel {mm}
{
    \mathinner{\frac{\partial #1}{\partial #2}}
}
\NewDocumentCommand \delnachdelsq {mm}
{
    \mathinner{\frac{\partial^2 #1}{\partial #2 ^2}}
}
\NewDocumentCommand \dnachd {mm}
{
    \mathinner{\frac{\symup{d} #1}{\symup{d} #2}}
}

\NewDocumentCommand \deln {mmm}
{
    \mathinner{\frac{\partial^#3 #1}{\partial #2 ^#3}}
}

\NewDocumentCommand \leftside {m}
{
\flushleft{#1\;}\justifying
}


\NewDocumentCommand \dif {m}
{
    \,\mathinner{\symup{d} #1}
}

\NewDocumentCommand \zz {}
{
    \mathinner{\mathrm{Z\kern-.3em\raise-0.5ex\hbox{Z}}}
}

\ExplSyntaxOff



% Unterstützung für Links und PDF Metadaten
\usepackage[unicode,pdfusetitle]{hyperref}
\usepackage{bookmark}

%\setmathfont{XITS Math}[range={scr,bfcal}]
%\setmathfont{XITS Math}[range={cal,bfcal}, StylisticSet=1]
\sisetup{math-micro=\text{\mu},text-micro=\mu}

\begin{document}
    \pagenumbering{roman}

    \vspace{2cm}
    
    \title{Measuring the Crab Nebula energy spectrum using FACT}
    
    \vspace{1cm}
    
    \author{
        Luca Di Bella\\
        \texorpdfstring{\href{mailto:luca.dibella@tu-dortmund.de}{luca.dibella@tu-dortmund.de}\and}{,}
        Luca Fiedler\\
        \texorpdfstring{\href{mailto:luca.fiedler@tu-dortmund.de}{luca.fiedler@tu-dortmund.de}}{}
    }
    
    \vspace{1cm}
    
    %\date{Durchführung:  \\ Abgabe:  \vspace{-4ex}}
    
    \maketitle
    \thispagestyle{empty}
    
    \vfill
    
    \begin{center}
        Technische Universität Dortmund\\
        Advanced Particle Physics Lab Course
    \end{center}
    
    \newpage
    \justifying
    \tableofcontents
    \newpage
    \pagenumbering{arabic}
    \clearpage
    \setcounter{page}{1}

% -- Zielsetzung -- %
    \section{Introduction}
    The First G-APD Cherenkov Tesescope (FACT) is a ground based telescope which utilizes Silicon Photomultipliers (aka. G-APTs, hence the name) to measure extensive air showers caused by relativistic Particles in the Atmosphere.
    This technique of measuring astronomic radiation offers various advantages over more direct measurement methods.
    Prominent targets of such measurements are gamma emitters such as super nova remnants such as the crab nebula and active galactic nuclei of various distant galaxies.

    \subsection{FACT}
    Fact is operational since 2011 and is positioned at the Observatorio Roque de los Muchachos on the Canary island of La Palma.
    Since beginning of operation, the telescope has collectet data on the gamma-ray emissions of varoius point of interest and passed the data on to the computer analysis part of the observation.
    
%
% -- Theorie -- %
%
    \section{Theory}


%
% -- Aufbau -- %
%
    \section{Experimental Setup}

%
% -- Durchführung -- %
%
    \section{Execution/Procedure}
    
%
% -- Auswertung -- %
%
    \section{Evaluation}


%
% -- Diskussion -- %
%
    \section{Discussion}

\newpage
\printbibliography
\newpage
\end{document}
