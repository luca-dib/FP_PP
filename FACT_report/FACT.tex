\documentclass[captions=tableheading,parskip=half, bibliography=totoc]{scrartcl} % KOMA-Script Dokumentenklasse Article [titlepage=firstiscover]

% UTF8 input encoding support
\usepackage[utf8]{inputenc}

% scrhack
\usepackage{scrhack}

% Warnung, falls noch einmal kompiliert werden muss
\usepackage[aux]{rerunfilecheck}

% Paket für Schriftarteinstellung, muss immer geladen werden
\usepackage{fontspec}

% Deutsche Spracheinstellungen, wichtig z. B. für korrekte Trennung
\usepackage[main=nenglish]{babel}
\usepackage[autostyle]{csquotes}

\usepackage{ragged2e}
\usepackage{pdfpages}
\usepackage{pdflscape}
\usepackage{rotating}

% adds Blindtext
\usepackage{blindtext}
\usepackage{longtable}

% Math and Science extensions
\usepackage{amsmath, nccmath}
\usepackage{amssymb}
\usepackage{mathtools}
\DeclarePairedDelimiter\abs{\lvert \,}{\rvert}
\DeclarePairedDelimiter{\bra}{\langle}{\rvert}
\DeclarePairedDelimiter{\ket}{\lvert}{\rangle}
\DeclarePairedDelimiterX{\braket}[2]{\langle}{\rangle}{
    #1 \delimsize| #2
    }

\usepackage[math-style=ISO,bold-style=ISO,sans-style=italic,nabla=upright,partial=upright]{unicode-math}
\usepackage[locale=DE, uncertainty-mode = separate]{siunitx}  %[per-mode=symbol-or-fraction]
\sisetup{math-micro=\text{$\symup{\mu}$},text-micro=$\symup{\mu}$}
\usepackage[version=4, math-greek=default, text-greek=default]{mhchem}
\DeclareSIUnit\Div{DIV}
%
\usepackage{graphicx}
\usepackage{grffile}
\usepackage{float}
\floatplacement{figure}{htbp}
\usepackage[labelfont=bf]{caption}
\usepackage[width=0.75\textwidth]{subcaption}

% Bibliography extensions
\usepackage{booktabs}
\usepackage[backend=biber]{biblatex}
\addbibresource{../lit.bib}

\usepackage{microtype}
\usepackage{xfrac}
\usepackage[shortcuts]{extdash}
\usepackage{expl3}
\usepackage{xparse}
\usepackage{mleftright}
\usepackage[shortcuts]{extdash}

\NewDocumentCommand \delnachdel {mm}
{
    \mathinner{\frac{\partial #1}{\partial #2}}
}
\NewDocumentCommand \delnachdelsq {mm}
{
    \mathinner{\frac{\partial^2 #1}{\partial #2 ^2}}
}
\NewDocumentCommand \dnachd {mm}
{
    \mathinner{\frac{\symup{d} #1}{\symup{d} #2}}
}

\NewDocumentCommand \deln {mmm}
{
    \mathinner{\frac{\partial^#3 #1}{\partial #2 ^#3}}
}

\NewDocumentCommand \leftside {m}
{
\flushleft{#1\;}\justifying
}


\NewDocumentCommand \dif {m}
{
    \,\mathinner{\symup{d} #1}
}

\NewDocumentCommand \zz {}
{
    \mathinner{\mathrm{Z\kern-.3em\raise-0.5ex\hbox{Z}}}
}

\ExplSyntaxOff



% Unterstützung für Links und PDF Metadaten
\usepackage[unicode,pdfusetitle]{hyperref}
\usepackage{bookmark}

%\setmathfont{XITS Math}[range={scr,bfcal}]
%\setmathfont{XITS Math}[range={cal,bfcal}, StylisticSet=1]
\sisetup{math-micro=\text{\mu},text-micro=\mu}

\begin{document}
    \pagenumbering{roman}

    \vspace{2cm}
    
    \title{Measuring the Crab Nebula energy spectrum using FACT}
    
    \vspace{1cm}
    
    \author{
        Luca Di Bella\\
        \texorpdfstring{\href{mailto:luca.dibella@tu-dortmund.de}{luca.dibella@tu-dortmund.de}\and}{,}
        Luca Fiedler\\
        \texorpdfstring{\href{mailto:luca.fiedler@tu-dortmund.de}{luca.fiedler@tu-dortmund.de}}{}
    }
    
    \vspace{1cm}
    
    %\date{Durchführung:  \\ Abgabe:  \vspace{-4ex}}
    
    \maketitle
    \thispagestyle{empty}
    
    \vfill
    
    \begin{center}
        Technische Universität Dortmund\\
        Advanced Particle Physics Lab Course
    \end{center}
    
    \newpage
    \justifying
    \tableofcontents
    \newpage
    \pagenumbering{arabic}
    \clearpage
    \setcounter{page}{1}

% -- Zielsetzung -- %
\section{Introduction}
The First G-APD Cherenkov Tesescope (FACT) is a ground based telescope which utilizes Silicon Photomultipliers (SiPMs) and more specifically Geiger Avalance Photodiodes (G-APDs), to measure Extensive Air Showers (EAS) caused by relativistic particles interacting with the atmosphere.
%This technique of measuring astronomic radiation offers various advantages over more direct measurement methods like FERMI.
Prominent targets of such measurements are gamma-ray emitters such as Super Nova Remnants (SNR) and active galactic nuclei of various distant galaxies.
In this report the energy spectrum of the crab nebula is analyzed using measurements taken with FACT.

\subsection{FACT}
FACT is operational since 2011 and is positioned at the Observatorio Roque de los Muchachos on the Canary island of La Palma.    
Similarly to other ground stationed gamma-ray telescopes such as MAGIC or CTAs telescopes, FACT measures the Cherenkov-light emitted by secondary particles in the atmosphere which are produced when the initial ray interacts with the atmosphere.
When a highly energetic gamma-ray interacts with particles in the atmosphere, electron-positron pairs are created which travel at relativistic speeds thereby emitting light via Cherenkov Radiation.
The amplitude of this light is dim compared to daylight and therefore observations can only be performed during night time.

Since the beginning of its operation, the telescope has collected data on the gamma-ray emissions of varoius points of interest and passed on the data to the computer analysis part of the observation.
Here, the raw information from observations is converted into a more managable form for further analysis.
Various calculations are performed, including the calibrations of each pixels readout in the cameras detector, including finding the mean time and intensity of the individual pixel.
Using these values, the background noise is removed from the pixels containing cherenkov light and the resulting shape in the camera view is used to determine the Hillas parameters including the estimated energy of the original ray.

\subsection{Simulated Data}
In the later steps of this analysis, reference data is required to the detector response which represents the imperfect measurement of an event.
This process involves simulated data created with the CORSIKA program \cite{corsika} by computing a digital representation of particles that are involved in the observation.
Then, the CERES software simulates how FACT would respond to these particles.
As the result of this method, samples of observations are given where the true properties of the initial particle are known.
Therefore, in the process of unfolding, data of real observations can be corrected using the calculations presented in the following chapter. 
    %
    % -- Theorie -- %
    %
    \section{Theory}

    The energy spectrum is given by the probability distribution function (PDF) $f(E)$ which represents the distribution of energy of individual particles.
    For experimental data, this continueous distribution can not be measured diectly but represented by a histogram in the form of an n-dimensional vector X containing binned even counts.
    Real measurements diverge from the true values of $X$ and are therefore the different vector $Y$ of dimension m.
    Connecting these two we find the migration matix $A$ representing the probabilities of one count to migrate from one energy bin to another:
    \begin{equation}
        y = A \cdot x
        \label{eqn:folding}
    \end{equation} 
    This discrete equation of vectors and a matrix represents the continueous Fredholm equation for folding:
    \begin{equation}
        g(y)= \int A(x,y)f(x) \symup(d)x 
    \end{equation}
    Here, $g(y)$ is the true pdf, $A(x,y)$ the folding function, $f(x)$ the distorted pdf corresponding to the true energy distribution, migration matrix, and measured energy respectively.

    The migration matrix is normalzed and can be inverted if its dimensions n and m are equal:
    \begin{equation}
        x = A^{-1}y
        \label{eqn:unfolding}
    \end{equation}
    If $n \neq m$, this is still possible by using a pseudoinverse $A^{-1}$ by means of singular value decomposition (SVD) for exapmle.
   
    Alternatively to SVD, the pseudoinverse matix $A$ can also be determined using minimization algorithms.
    One such algorithm is the Poisson-likelihood optimization under the assumption that both $f$ and $g$ follow a Poisson distribution.

    Under this condition, the bin populations 
    \begin{equation}
        P(g_i) = P(g_i,\lambda_i)
    \end{equation}
    with 
    \begin{equation}
        \lambda = A \cdot f
    \end{equation}
    resulting in the likelihood function
    \begin{equation}
        \mathcal{L} = \prod_{i=1}^M P(g_i,\lambda)
    \end{equation}
    which can be minimized as follows:
    \begin{equation}
        \text{argmin}(- \ln \mathcal{L})=\text{argmin}\left(\sum_{i=1}^M \ln(g_i \!)-g_i \cdot \ln\lambda_i+\lambda_i\right)
        \label{eq:PoissonLikelihood}
    \end{equation}
    This computation is done numerically.

    In energy spectrum measurements the effective detector area $A_\text{eff}$ is a function of energy and needs to be calculated for each energy bin individually.
    This effective area contributes to the deviations between the measured events and existing events.
    Thanks to the simulated data, $A_{\text{eff},i}$ for each bin $i$ can be calculated as follows:
    \begin{equation}
        A_{\text{eff},i} = \frac{N_{\text{detected},i}}{N_{\text{simulated},i}} \cdot A
        \label{eq:acceptance}
    \end{equation}
    With the sumulated number of events $N_{\text{simulated},i}$ and the number of events detected by the telescope $N_{\text{detected},i}$.

    The flux $\Phi_i$ in each bin can be calculated as follows:
    \begin{equation}
        \Phi_i = \frac{\hat{f}_i}{A_{\text{eff},i} \cdot \Delta E_i \cdot t_\text{obs}}
        \label{eq:flux}
    \end{equation}
    With the bin energy range $\Delta E_i$ and the observation time $t_\text{obs}$
    
    For the significance $S$ of a whole measurement the Likelihood-Ratio Test taken from Li and Ma \cite{significance} can be used:
    \begin{equation}
        S = \sqrt{2} \cdot \sqrt{N_\text{on} \ln\left( \frac{1 + \alpha}{\alpha} \left( \frac{N_\text{on}}{N_\text{on} + N_\text{off}} \right) \right) + N_\text{off} \ln\left( (1 + \alpha) \left( \frac{N_\text{off}}{N_\text{on} + N_\text{off}} \right) \right) }
        \label{eq:significance}
    \end{equation}
    Here, the $N_\text{on}$ represent the envent counts of a given source, $N_\text{off}$ the reference background noise, and $\alpha$ is the ratio between the On- to the Off-regions.
%
% -- Auswertung -- %
%
    \section{Analysis}
        For the analysis of the Crab Nebula energy spectrum three datasets from the open FACT dataset are used.\cite{FACTData}
        \newline
        One constitutes real measured data of $\SI{17.7}{\hour}$ of observation time of the Crab Nebula with FACT.
        The two others are a set of reconstructed simulated gamma-ray events and the CORSIKA headers for these, which contain information about all the simulated air showers.

        This analysis was conducted using the programming language python and its modules numpy\cite{numpy}, scipy\cite{scipy} and pandas\cite{pandas}.
        For data extraction from the bulk event files the module pyfact was utilized and for creation of the plots the matplotlib\cite{matplotlib} module was used.
        \subsection{Theta-Square-Plot and Significance}
            First a Theta-Square plot is created for the real Crab Nebula measurements.
            For this all events which the Random Forest Classifier classifies as gamma-ray events with a confidence of $\geq \num{0.8}$ are histogrammed.
            The resulting plot is shown in Figure \ref{fig:theta_sq}.
            \begin{figure}
                \centering
                \includegraphics[width=\textwidth]{build/theta_sq.pdf}
                \caption{
                    Theta-Square plot of data from $\SI{17.5}{\hour}$ of observation time of the crab nebula.
                    The event counts for the On-position and the five Off-positions are histogrammed into 20 bins.
                    For clarity the combined counts and errorbars of the Off-position measurements are divided by the number of positions.
                }
                \label{fig:theta_sq}
            \end{figure}
            
            Based on this the significance of detection can be calculated with the likelihood-ratio test due to Li \& Ma \cite{} as shown in equation \eqref{eq:significance}.
            The cut of $\theta^2 = \num{0.025}$ results in values of $N_\text{on} = 2582$ and $N_\text{off} = 8916$ and with a ratio of one to five of On- to Off-positions $\alpha = \num{0.2}$ results.
            With these a significance of
            \begin{equation}
                S = \num{15.9} \; \sigma
            \end{equation}
            is calculated.
            

        
        \subsection{Energy Migration}
            Next the detector response matrix $A$, which in this case is the so-called energy migration matrix of the Random Forest Regressor, is determined.
            To do this the simulated gamma-ray event data has to be used.
            This is because knowledge of not only the predicted, but also the true energies of the events is needed.
            
            The true and predicted energies are histogrammed in a two-dimensional histogram.
            The bins of this histogram constitute the matrix in question, though it is not normalized yet.
            To do this the matrix elements in each row are divided by the sum of the elements in that row.
        \subsection{Unfolding of Crab Nebula Data}
            Using the detector response matrix real data can now be unfolded.
            There are many different methods of unfolding.
            Here only two of these will be explored, those being naive SVD unfolding as shown in equation \eqref{eqn:unfolding} as well as Poisson-likelihood unfolding as shown in equation \eqref{eq:PoissonLikelihood}.

            Applying these methods to the energy spectrum of the crab nebula data gives the unfolded histograms as shown in figure \ref{fig:unfolding}.
            \begin{figure}
                \centering
                \includegraphics[width=\textwidth]{build/unfolding.pdf}
                \caption{
                    The measured and unfolded crab nebula data recorded with FACT.
                    Shown in red is the measured energy spectrum.
                    Shown in green (SVD method) and blue (Poisson-likelihood method) are the unfolded energy spectra.
                    For both the data is binned logarithmically in an energy range of $\SI{500}{\giga\electronvolt} \leq E \leq \SI{15}{\tera\electronvolt}$, with under and overflow bins respectively as required.
                }
                \label{fig:unfolding}
            \end{figure}

        \subsection{Flux Calculation for Crab Nebula}
            To calculate the flux from the energy spectrum the effective area of the perfect detector to correct for the limited acceptance of the real detector.
            This is done according to equation \eqref{eq:acceptance}, for each bin in the real energy.

            Using this, the width of the energy bins $\Delta E_i$ and the observation time $t_\text{obs} \approx \SI{17.7}{\hour}$, the flux can now be calculated using equation \eqref{eq:flux}.
            The result of this can be seen in figure \ref{fig:flux}.
            \begin{figure}
                \centering
                \includegraphics[width=\textwidth]{build/flux.pdf}
                \caption{
                    The flux of the crab nebula as calculated from the FACT data.
                    Shown in green and blue is the flux calculated with the SVD and Poisson-likelihood unfolded energy spectra respectively.
                    Additionally the fit curves calculated from data from the MAGIC\cite{MAGIC} and HEGRA\cite{HEGRA} gamma-ray observatories are shown for comparison.
                }
                \label{fig:flux}
            \end{figure}


%
% -- Diskussion -- %
%
    \section{Results and Discussion}
        \subsection{Theta-Square-Plot and Significance}
            In the Theta-Square-Plot seen in figure \ref{fig:theta_sq} the counts of the Off-positions show a noticable background.
            As is expected of an (at the very least locally) isotropic background the counts are spread over the squared angle in a uniform distribution.
            This measurement shows the importance of running the telescope in the \enquote{Wobble} observation mode for obtining a clean background sample measurement.

            Regarding the significance, traditionally in the field of physics a detection with significance of $3\:\sigma$ is considered evidence, while a detection of $5\:\sigma$ is considered a discovery.
            With a significance of $\num{15.9}\:\sigma$ this measurement would be considered a discovery.
            This shows very clearly that the crab nebula is indeed a source of gamma-ray emissions.

        \subsection{Energy Migration}
        \subsection{Unfolding of Crab Nebula Data}
            The two methods used for unfolding differ only very slightly in their result.
            SVD unfolding gives almost negligable errors, so much so that it was necessary to magnify them thousandfold to make them visible in figure \ref{fig:unfolding}, while Poisson-Likelihood unfolding yields comparable errors to the measured spectrum.
            
            The measured data contained no event which was measured below $\SI{500}{\giga\electronvolt}$, although unfolding shows a not insignificant amount of events here.
            This can be explained with the energy sensitivity of FACT.
            The telescope is very likely insensitive for gamma-rays with energies below this threshold, which results in no measured events in this region.
        \subsection{Flux Calculation for Crab Nebula}
            Comparing the flux calculated with FACT to the fits of MAGIC and HEGRA shows a clear overestimation for energies above $\SI{1}{\tera\electronvolt}$, with deviations up to more than one order of magnitude in the high end of the observed energies.
            The flux for the different methods of unfolding the energy spectrum show almost negligable difference, thus the discrepancy cannot be explained by this.

            Part of a more likely explaination is simply a lack of data.
            The data used in this experiment is from an observation time of only around $\SI{17.7}{\hour}$, whereas the data from HEGRA comes from almost $\SI{400}{\hour}$ of observation time and the MAGIC data from $\SI{69}{\hour}$.
            The discrepancy in the results might be due to this, although the low bin count to maximize events per bin should alleviate statistical error due to low event counts.

            Another possible explaination is the operating mode of this telescope.
            The FACT telescope is a single gamma-ray telescope, whereas the MAGIC and HEGRA telescopes work in stereoscopic mode, where two or more telescopes observe the same target simultaneously.
            HEGRA used 5 telescopes simultaneously, while MAGIC uses two.
            This operating mode not only improves direction measurement, but also offers background rejection since it is unlikely that multiple telescopes measure background radiation simultaneously.
            

\newpage
\printbibliography
%\newpage
\end{document}